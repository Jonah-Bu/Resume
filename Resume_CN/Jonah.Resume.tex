\documentclass{resume} % Use the custom resume.cls style

\usepackage[left=0.75in,top=0.6in,right=0.75in,bottom=0.6in]{geometry} % Document margins
\usepackage[T1]{fontenc}
\usepackage{lmodern}
\usepackage{graphicx}        % standard LaTeX graphics tool
\usepackage[colorlinks, linkcolor=blue, CJKbookmarks=true]{hyperref}
\usepackage{CJK}




\begin{CJK}{UTF8}{gbsn}


%\name{卜晶礻韦} % Your name
%\address{86 Overland Road \quad \\ \quad Waltham, MA 02451} % Your address
%% \address{86 Overland Road \\ Waltham, MA 02451} % Your secondary addess (optional)
%\address{15921913396 \quad \\ \quad Bu.Jonah@gmail.com} % Your phone number and email
\begin{document}
\begin{figure}
	\centering
	\includegraphics[scale=1.2]{images/name.pdf}
\end{figure}
\vspace{-60pt}
\begin{center}
	15921913396 \quad \\ \quad Bu.Jonah@gmail.com
\end{center}

%----------------------------------------------------------------------------------------
%	EDUCATION SECTION
%----------------------------------------------------------------------------------------
\
\begin{rSection}{教育}
{\bf 上海交通大学} \hfill {2010.09 - 2013.03} \\ 
硕士, 生物医学工程 \\
课题: Retina神经网络编码机制\\

\vspace{-10pt}
{\bf 上海交通大学} \hfill {2006.09 - 2010.06} \\ 
本科, 生物医学工程 \\

\end{rSection}

%----------------------------------------------------------------------------------------
%	WORK EXPERIENCE SECTION
%----------------------------------------------------------------------------------------
\vspace{-10pt}
\begin{rSection}{经历}
\begin{rSubsection}{飞利浦心血管院后数据分析系统}{2015.8 -
    现在}{Data Scientist}{飞利浦中国研究院}
\item 用户行为分析与个性化推荐:
\begin{enumerate}
\item 利用系统所收集的用户信息(包括年龄、性别、教育程度、家庭情况等所有与用户相关的信息)和系统所记录的用户行为(包括用药依从性、检查录入情况、在系统中的行为等等),判断哪些用户是该系统的潜在客户,以及哪些用户在使用该系统后会获得医学收益
\item 使用stacking(RF+GBDT+LR)对用户进行分类
\item 建立知识库,根据不同的用户分类进行个性化推荐
\end{enumerate}
\end{rSubsection}

%------------------------------------------------
\begin{rSubsection}{个性化推荐知识库}{2015.07 - 2015.12}{Data Scientist}{飞利浦中国研究院}
\item  为个性化推荐系统建立知识库和规则匹配系统:
\begin{enumerate}
\item 提出一个多层知识抽取框架,将原始的以自然语言描述的“知识”逐层抽象成为计算机可执行的语句 
\item 使用wiki作为知识管理平台
\item 编写爬虫将wiki页面上的知识抽取并写入MongoDB
\item 使用Drools管理和执行匹配规则 
\end{enumerate}
\end{rSubsection}
%------------------------------------------------

\begin{rSubsection}{高风险孕妇预测模型}{2014.06 - 2015.07}{Data Scientist}{飞利浦中国研究院}
\item 基于移动平台的孕妇风险监测系统(Mobile Obstetrical Monitoring):
\begin{enumerate}
\item 使用Logistic Regression(LR)和Random Forests(RF)算法建立孕妇孕期出现高危高血压以及 preeclampsia疾病的预测模型
\item  对数据进行特征选择,使用包括在LR中加入正则化惩罚项(Lasso),和在RF中使用out of bag (OOB)估计变量重要性(variable importance)
\item 试验不同的梯度下降(GD)算法对结果的影响,包括batch GD,Momentum,Adam,以及BFGS
\item 在android平台开发一个原型,用于临床医生对该模型进行验证
\end{enumerate}
\end{rSubsection}
%------------------------------------------------

\begin{rSubsection}{个人健康管理与个性化推荐系统}{2014.01 - 2014.12}{
		Data Scientist}{飞利浦中国研究院}
\item 个人健康管理系统(Personal Health Management Solution):
\begin{enumerate}
\item 使用Logistic Regression(LR)和支持向量机(SVM)算法建立四年期高血压、八年期糖尿病和十年期心血管疾病的风险预测模型
\item 利用不断得到的新数据,尝试建立了预测模型的在线版本(online LR)
\end{enumerate}
\end{rSubsection}
%------------------------------------------------

\begin{rSubsection}{智能检测套件}{2013.04 - 2013.12}{Data Scientist}{飞利浦中国研究院}
\item 智能建议模块的算法设计(临床决策支持):
\begin{enumerate}
\item 进行文献研究(literature study),利用文献中已报导的模型(包括Weibull Regression和Cox Regression)对患者进行风险预测
\item 建立知识库,建立风险预测结果和个性化建议之间的映射关系
\end{enumerate}
\end{rSubsection}

%------------------------------------------------

\end{rSection}

%----------------------------------------------------------------------------------------
%	TECHNICAL STRENGTHS SECTION
%----------------------------------------------------------------------------------------

\begin{rSection}{技能}

\begin{itemize}
	\item 熟练掌握各种机器学习算法,包括回归和分类(LR,LAR,DT,SVM等),聚类(kNN,kMeans),以及集成学习算法(boosting,bagging和stacking)
	\item 理解机器学习理论,包括PAC学习,VC理论,以及Rademacher复杂度
	\item 掌握深度学习算法,包括RBM,DBN等
	\item 掌握各种优化算法,包括梯度下降及其变种(BGD,SGD,Momentum,Adam等),牛顿法 (newton, BFGS), 最速下降(SD, CG),Markov Chain Monte Carlo,Variational Bayesian Inference 
	\item 掌握各种特征选择(feature selection)方法,包括filter,wrapper,embedded
	\item 了解Hadoop,MapReduce,Spark
	\item 熟练掌握Python,Matlab,R
	\item 掌握Java, C++
\end{itemize}

\end{rSection}

%----------------------------------------------------------------------------------------
%	PUBLICATION SECTION
%----------------------------------------------------------------------------------------

\begin{rSection}{发表}

{\bf 期刊}
\begin{itemize}\small
\item {\bf Jingyi Bu}, Hao Li, Hai-Qing Gong, Pei-Ji Liang, Pu-Ming Zhang. ``Gap junction permeability modulated by dopamine exerts effects on spatial and temporal correlation of retinal ganglion cells’ firing activities.'' in {\em Journal of Computational Neuroscience}, 2013. (SCI indexed, IF = 2.51) 
\end{itemize}


{\bf 国际会议}
\begin{itemize}\small
\item {\bf Jingyi Bu}, Ning Lan.  ``An Improved Multi-Channel Cortical Recording And Stimulation System.'' {\em International Convention on Rehabilitation Engineering \& Assistive Technology}, p. 98-101, 2010. (EI indexed)
\end{itemize}

\end{rSection}

%----------------------------------------------------------------------------------------

%----------------------------------------------------------------------------------------
%	PATENT SECTION
%----------------------------------------------------------------------------------------

\begin{rSection}{专利}

\begin{itemize}
\item  Wang Jin, {\bf Bu Jingyi}.  \href{http://www.google.com/patents/WO2015052609A1?cl=en}{``An Apparatus and Method for Evaluating Multichannel ECG Signals''} WO2015052609A1.16/04/2015
\end{itemize}

\end{rSection}

%----------------------------------------------------------------------------------------



%----------------------------------------------------------------------------------------
%	EXAMPLE SECTION
%----------------------------------------------------------------------------------------

%\begin{rSection}{Section Name}

%Section content\ldots

%\end{rSection}

%----------------------------------------------------------------------------------------
% \clearpage
\end{CJK}
\end{document}
